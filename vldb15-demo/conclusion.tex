\section{Conclusion}
The prevalence of dirty data presents a fundamental obstacle to modern data-driven applications.
We introduced \sys, a system designed to support the iterative development of data cleaning workflows.
\sys allows the user to construct, adapt, and optimize data cleaning workflows with automated parameter recommendations.
Our main contribution is a separation of logical data cleaning operators and their physical implementations (e.g., rules, learning, or crowdsourced).
In our demo, we illustrate how \sys can be used to clean three real datasets, Zagat, Yelp, and Iowa Alcohol Sales, with different physical implementations of the same logical Extraction and Entity Resolution workflow.
The physical implementations, while the same at a logical level, have vastly different cleaning accuracies and our system aids the user in selecting the best options.
In our initial code release, we include the core mechanisms for declarative specification of the data cleaning pipeline, operator API design, and include support for, and implementations of, multiple classes of physical data cleaning operators.

\vspace{0.5em}

\textbf{This material is based upon work supported by the National Science Foundation Graduate Research Fellowship under Grant No. DGE 1106400. This research is also supported in part by NSF CISE Expeditions Award CCF-1139158, LBNL Award 7076018, and DARPA XData Award FA8750-12-2-0331, and gifts from Amazon Web Services, Google, SAP, The Thomas and Stacey Siebel Foundation, Adatao, Adobe, Apple, Inc., Blue Goji, Bosch, C3Energy, Cisco, Cray, Cloudera, EMC2, Ericsson, Facebook, Guavus, HP, Huawei, Informatica, Intel, Microsoft, NetApp, Pivotal, Samsung, Schlumberger, Splunk, Virdata and VMware.}